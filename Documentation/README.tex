\documentclass[10pt,a4paper]{report}
\usepackage[utf8]{inputenc}
\usepackage{amsmath}
\usepackage{amsfonts}
\usepackage{amssymb}
\usepackage{xcolor}
\usepackage{hyperref}

\author{GG}
\title{README}
\date{}
\begin{document}
\maketitle
\section*{About Encoding Simulator: }
Encoding Simulator it is a program written in Python. It takes a bit stream from the user and 
then it encodes the stream while plotting an ASCII - based voltage graph. The methods that the program
uses to encode the stream are the following:
\begin{itemize}
	\item{NRZ-L}
	\item{NRZ-I}
	\item{Bipolar AMI}
	\item{Pseudoternary}
	\item{Manchester}
	\item{Diff. Manchester}
\end{itemize}

\section*{How to use: }
The program is tested to be running on GNU/Linux and Windows 10. I have no indication that the program will not be running
on Mac since python can be installed on Mac, but i have never tested it. 
\begin{itemize}
	\item{Linux:} In the unlikely scenario that your linux installation is not shipped with python you can install python3x on any           distribution either using synaptic or pacman (it is in the core repositories)
	and then run the program using your favorite terminal by issue this command "python3 encoder.py". The program 
	asks you to give a bit stream and then to choose a method of encoding.
	\item{Windows 10:} It is highly unlikely that your Windows system shipped with Python already installed so you may have to 
	download the python from this \href{https://www.python.org/}{\textcolor{blue}{URL}}
	\begin{itemize}
		\item Open a browser window and navigate to the Download page for Windows at python.org.
		\item On the heading at the top of the site it says Downloads. Click Windows.
		\item Find the newest version of python3. At the point of writing this,  it is Python3.8.1 and click on download the executable installer.
		\item Once the installer is downloaded run it and install it on your machine.
		\item Run the program by issue this command "python encoder.py" in the cmd 
	\end{itemize}
\end{itemize}
\section*{Legal:} Encoding Simulator is licensed under the terms of }{\textcolor{blue}{URL}}.

\end{document}
